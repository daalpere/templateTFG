\documentclass[12pt,a4paper,twoside]{book}
% habilitamos Arial en el documento
\usepackage{mathptmx}
\usepackage{anyfontsize}
\usepackage{t1enc}
\usepackage{fontspec}
\usepackage{url}

\setromanfont[
BoldFont=arialbd.ttf,
ItalicFont=ariali.ttf,
BoldItalicFont=arialbi.ttf,
]{arial.ttf}

%Habilitamos el TOC hasta subsubsection
\setcounter{tocdepth}{3}
\setcounter{secnumdepth}{3}

\newcommand\tab[1][1cm]{\hspace*{#1}}

\usepackage{setspace}
\renewcommand{\baselinestretch}{1.5}

\usepackage{afterpage}
\usepackage{ragged2e}
\usepackage[none]{hyphenat}
\usepackage{xpatch}
\usepackage{blindtext}
\usepackage{sectsty}
\usepackage{appendix}


%\def\UrlBreaks{\do\/\do-}
\usepackage{hyperref}
%\usepackage{breakurl}
%\usepackage[breaklinks]{hyperref} % Referencias e índices son enlaces al contenido. La opción hidelinks hace que se vean como texto normal


\usepackage{newclude} % Permite usar \include* que es igual que \include pero sin saltos de página adicionales
%\includeonly{cap01} % con este comando solo se incluirán los archivos mencionados haciendo que la compilación sea mucho más rápida. Referencias a otros capítulos quedarán rotas, se recomienda una compilación limpia cada vez que se cambie esta línea (es decir, borrar ficheros intermedios cuando se descomenta/comenta/cambia el argumento)

\usepackage[top=3.0cm, bottom=2.0cm,left= 2.5cm,right=2cm]{geometry}

\usepackage{csquotes} % Paquete recomendado por biblatex cuando se usa babel

\usepackage{graphicx} % Gráficos y colores
\usepackage{amsmath,amssymb}
\usepackage{float}
\usepackage{changepage}
\usepackage{subcaption}

\usepackage[table]{xcolor} % Necesaria la opción table para usar filas de colores en las tablas.
\usepackage{lipsum} % Para usar texto de relleno, sirve para tener una idea de cómo queda

\usepackage{booktabs} % para los comandos toprule, midrule, cmidrule y bottomrule: líneas horizontales de las tablas
\usepackage{threeparttable} % Entorno para las notas a pie de tabla

%--- Paquete algorithm y personalizaciones ---%
\usepackage{algorithm,algorithmic}
\floatname{algorithm}{Algoritmo}
\renewcommand{\listalgorithmname}{\'INDICE DE ALGORITMOS}

\usepackage{multirow}

\usepackage{tikz}

%--- Formato de títulos ---
\usepackage{titlesec} % Librería para cambiar el formato de \chapter, \section, ...
\titleformat{\chapter}[display]
{\bfseries\centering\fontsize{16pt}{16}\selectfont}{\centering\chaptertitlename \thechapter}{24pt}{}
\titlespacing*{\chapter}
{0pt}{30pt}{40pt}
\renewcommand{\chaptertitlename}{\fontsize{24pt}{24}\selectfont CAP\'ITULO }
\newpagestyle{mystyle}{\setfoot[][\thepage][]{}{\thepage}{}}
\pagestyle{mystyle}
\titleformat{\paragraph}[runin]{\normalfont\normalsize\itshape}{\theparagraph}{1.5em}{} % formato de \paragraph: fuente y tamaño normales y cursiva. 1em de separación
\titleformat{\subparagraph}[runin]{\normalfont\normalsize\itshape}{\thesubparagraph}{1.5em}{} % formato de \subparagraph: fuente y tamaño normales y cursiva. 1em de separación

%Renombrar los epigrafos
\renewcommand{\tablename}{Tabla}
\renewcommand{\figurename}{Figura}



%--- AJUSTES MANUALES PARA CODIGOS ---%

\usepackage{listings}
\usepackage{color}

\definecolor{dkgreen}{rgb}{0,0.6,0}
\definecolor{gray}{rgb}{0.5,0.5,0.5}
\definecolor{mauve}{rgb}{0.58,0,0.82}

\lstset{frame=tb,
  language=Java,
  aboveskip=3mm,
  belowskip=3mm,
  showstringspaces=false,
  columns=flexible,
  basicstyle={\footnotesize\ttfamily\justifying},
  numbers=none,
  numberstyle=\tiny\color{gray},
  keywordstyle=\color{blue},
  commentstyle=\color{dkgreen},
  stringstyle=\color{mauve},
  breaklines=true,
  breakatwhitespace=true,
  tabsize=3
 }
 
\renewcommand\maketitle{}


%%%%%%%%%%%%%%%%%%%%%%%%%%%%% DOCUMENTO %%%%%%%%%%%%%%%%%%%%%%%%%%%%%
\begin{document}
\sloppy 
\maketitle

%%%%%%%%%%%%%%%%%%%%%%%%%%%%% TÍTULO, RESUMEN, AGRADECIMIENTOS Y DEDICATORIA %%%%%%%%%%%%%%%%%%%%%%%%%%%%%
%	TÍTULO
% Página con el título, hay que entrar en title.tex para modificar: nombre del trabajo, autor y fecha

\include*{title}

% DEDICATORIA,AGRADECIMIENTO,DECLARACION_EXPRESA Y FIRMAS
\include*{adicionales}

\frontmatter

% RESUMEN
\include*{resumen}

%	ÍNDICE
%  INDICE GENERAL FORMATO
\renewcommand*{\contentsname}{\'INDICE GENERAL}
\tableofcontents  % indice de contenidos

% ABREVIATURAS
\include*{abreviaturas}

% SIMBOLOGIA
\include*{simbologia}

%	INDICE DE FIGURAS, TABLAS Y ALGORITMOS
\renewcommand*{\listfigurename}{\'INDICE DE FIGURAS}
\addcontentsline{toc}{chapter}{\listfigurename}
\listoffigures

\renewcommand*{\listtablename}{\'INDICE DE TABLAS}
\addcontentsline{toc}{chapter}{\listtablename}
\listoftables

% Si no hay muchos algoritmos no tiene mucho sentido esta lista
\addcontentsline{toc}{chapter}{\listalgorithmname}
\listofalgorithms 

%%%%%%%%%%%%%%%%%%%%%%%%%%%%% CUERPO DEL TFG %%%%%%%%%%%%%%%%%%%%%%%%%%%%%
\mainmatter

%reconfigurar el titulo de cada capitulo (DON't TOUCH)
\titleformat{\chapter}[display]
{\bfseries\filright\fontsize{16pt}{16}\selectfont }{\centering\chaptertitlename \thechapter}{24pt}{\thechapter. \,}

% TABLAS, FIGURAS, EXPRESIONES MATEMÁTICAS Y ALGORITMOS
\include*{cap01}
% CONTENIDOS DEL TFG
\include*{cap02}
% REFERENCIAS
\include*{cap03}
% CONCLUSIONES Y LÍNEAS FUTURAS DE TRABAJO
\include*{cap04}
% AGREGAR MAS CAPITULO
\include*{cap05}

%%%%%%%%%%%%%%%%%%%%%%%%%%%%% BACKMATTER %%%%%%%%%%%%%%%%%%%%%%%%%%%%%
\backmatter
\titleformat{\chapter}[display]
{\bfseries\centering\fontsize{24pt}{28}\selectfont}{\centering\chaptertitlename \thechapter}{24pt}{}
\titlespacing*{\chapter}
{0pt}{30pt}{40pt}
%%% Bibliografía bibname
\renewcommand{\bibname}{BIBLIOGRAF\'IA}
\addcontentsline{toc}{chapter}{BIBLIOGRAF\'IA} % Para que salga en el índice general
\bibliographystyle{ieeetr}
\bibliography{referencias}

%%%%%%%%%%%%%%%%%%%%%%%%%%%%% APÉNDICES %%%%%%%%%%%%%%%%%%%%%%%%%%%%%

\appendix
\renewcommand{\appendixtocname}{AP\'ENDICES}
\renewcommand{\appendixname}{AP\'ENDICES}
\renewcommand{\appendixpagename}{AP\'ENDICES}
\addappheadtotoc
\appendixpage

\include*{apendices}


\end{document}